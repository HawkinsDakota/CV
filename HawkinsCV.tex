% !TEX program = xelatex
\documentclass[a4paper,10pt]{article}

%A Few Useful Packages
\usepackage{marvosym}
\usepackage{fontspec} 					%for loading fonts
\usepackage{xunicode,xltxtra,url,parskip} 	%other packages for formatting
\usepackage{setspace, lipsum}	% packages for line spacing
\RequirePackage{color,graphicx}
\usepackage[usenames,dvipsnames]{xcolor}
\usepackage[big]{layaureo} 				%better formatting of the A4 page
% an alternative to Layaureo can be ** \usepackage{fullpage} **
\usepackage{supertabular} 				%for Grades
\usepackage{titlesec}					%custom \section
%Setup hyperref package, and colours for links
\usepackage{hyperref}
\definecolor{linkcolour}{rgb}{0,0.2,0.6}
\hypersetup{colorlinks,breaklinks,urlcolor=linkcolour, linkcolor=linkcolour}
%FONTS
\defaultfontfeatures{Mapping=tex-text}
\setmainfont{Helvetica} %Set font to Helvetica

%CV Sections inspired by:
%http://stefano.italians.nl/archives/26
\titleformat{\section}{\Large\scshape\raggedright}{}{0em}{}[\titlerule]
\titlespacing{\section}{0pt}{0pt}{0pt}

%Italian hyphenation for the word: ''corporations''
\hyphenation{im-pre-se}

%-------------WATERMARK TEST [**not part of a CV**]---------------
\usepackage[absolute]{textpos}

\setlength{\TPHorizModule}{30mm}
\setlength{\TPVertModule}{\TPHorizModule}
\textblockorigin{1mm}{0.65\paperheight}
\setlength{\parindent}{0pt}

%-------------Document Size -----------------------
\usepackage{geometry}
\geometry{ bmargin=0.25in, tmargin = 1in}

%--------------------BEGIN DOCUMENT----------------------
\begin{document}

\pagestyle{empty} % non-numbered pages

\font\fb=''[cmr10]'' %for use with \LaTeX command

%--------------------TITLE-------------
\par{\centering
		{\Huge Dakota \textsc{Hawkins}
	}\bigskip\par}

%--------------------SECTIONS-----------------------------------
%Section: Personal Data
\section{\color{linkcolour}{Contact}}

\begin{tabular}{rl}
    \textsc{Address:} & 36 Bellvista Road Apt. 24, Boston, MA \\
    \textsc{Phone:}   & (435)-764-5762 \\
    \textsc{e-mail:}  & \href{mailto:dyh0110@bu.edu}{dyh0110@bu.edu} \\
    \textsc{GitHub:}  & \href{https://github.com/dakota-hawkins}{https://github.com/dakota-hawkins}
\end{tabular}

%Section: Education
\section{\color{linkcolour}{Education}}
\begin{tabular}{rl}
\textsc{2016 -- Present} & Doctor of Philosophy, \textbf{Boston University}, Boston, MA \\
& Bioinformatics | Cynthia A. Bradham Laboratory \\

 \textsc{2010 -- 2015} & Bachelor of Science, \textbf{Westminster College}, Salt Lake City, UT\\
& \small\emph{cum laude} | Majors: Biology and Mathematics\\
&\normalsize \textsc{GPA}: 3.7
\end{tabular}

%Section: Work Experience at the top
\section{\color{linkcolour}{Work Experience}}
\begin{tabular}{r|p{11cm}}
 \textsc{2015 -- 2016}     & Pacific Northwest National Laboratory, Richland, WA \\
  	& \emph{Post Baccalaureate Research Assistant} \\
	& \footnotesize{Worked in the Applied Statistics and Computational Modeling
					group under the Computational and Statistical Analysis division.
					Research focused on bioinformatic-based projects such as
					analysis of -omics data and development of new quantitative
					tools to assist researchers.}\\
\multicolumn{2}{c}{} \\

\textsc{2013 -- 2015} & Westminster College, Salt Lake City, UT \\
 & \emph{QUARC Student Statistics Consultant}\\
 & \footnotesize{Helped develop quantitative reasoning on Westminster College Campus.
				 Responsibilities focused on aiding in statistical analysis for
				 local projects, teaching in-class lessons, and devoloping new
				 quantitative literacy courses for Westminster College}\\
\multicolumn{2}{c}{} \\
\end{tabular}

%Section: Research
\section{\color{linkcolour}{Research}}
\begin{tabular}{rp{10cm}}
\textsc{May 2017 -- Present} & Cynthia A. Bradham Laboratory at Boston University, Boston, MA \\
	& \footnotesize{Developing novel algorithms to identify shared cell-types across treatments
					in scRNAseq data, and to integrate spatial information from fluorescence
					imaging with high-throughput scRNAseq.} \\
\textsc{Jan. 2017 -- May 2017} & Paola Sebastiani Laboratory at Boston University, Boston, MA \\
	& \footnotesize{Performed eQTL analysis to establish tissue-specific biomarkers for 
	                Alzheimer's disease.} \\
\textsc{Sept. 2016 -- Dec. 2016} & Stefano Monti Laboratory at Boston University, Boston, MA \\
	& \footnotesize{Leveraged general linear models to determine cancer-specifc immune response in
	                tumor cells.} \\
\textsc{Jul. 2016 -- Sept. 2016} & James Galagan Laboratory at Boston University, Boston, MA \\
& \footnotesize{Conducted ChIP-Seq and RNA-Seq experiments to help map the transcriptional
	            regulatory network of \emph{E.\ coli}.} \\
\textsc{Mar. 2016 -- Jul. 2016} & Pacific Northwest National Laboratory, Richland, WA \\
& \footnotesize{Aided in protein-based stable isotope probing experiments by
                running analysis pipelines to  calculate labeling statistics.} \\
\textsc{Nov. 2015 -- Jul. 2016} & Pacific Northwest National Laboratory, Richland, WA \\
& \footnotesize{Provided statistical support to determine differences in -omic
				make-up of the fecal microbiome between successful and
				unsuccesful gastric bypass patients.} \\
\textsc{Jul. 2015 -- Feb. 2016} & Pacific Northwest National Laboratory, Richland, WA \\
& \footnotesize{Helped create and implement displays and algorithms to visualize
                and quantify shotgun proteomic data.} \\
 \textsc{2013 -- 2014} & Westminster College, Salt Lake City, UT\\
& \footnotesize{Developed novel program in Python for automating detection of
                singing on the nest in field recordings of Northern Mockingbirds.}\\
\textsc{2012 -- 2013} & Westminster College, Salt Lake City, UT\\
& \footnotesize{Collected field recordings of House Finch songs to compare urban
                and non-urban song dialects.}\\
\textsc{Jan. 2012 -- Jun. 2012} & University of Utah Health Care, Salt Lake City, UT
\\& \footnotesize{Aided in genetic analysis running reverse transcription and PCR analysis.}
\end{tabular}

%Section: Relevant Course work
\section{\color{linkcolour}{Relevant Course Work}}
\begin{tabular}{rl}
\textsc{Math} & Mathematical Biology (I \& II), Differential Equations, Mathematical Statistics,
\\ & Probability and Statistics, Applied Statistics, Statistics for the Life Sciences,
\\ & Networks, Abstract Algebra \\
\textsc{Science} & Genetics, Cell Biology, Organic Chemistry, Developmental Biology, Ecology \\
\textsc{Computing} & Scientific Computing, Intro to Data Structures
\end{tabular}

%Section: Languages
\section{\color{linkcolour}{Programming Languages}}
\begin{tabular}{rl}
	\textsc{Python:}& Used generally for data analysis, machine learning, and package development.\\
	& \small{\href{https://github.com/dakota-hawkins/yamada}{https://github.com/dakota-hawkins/yamada}}\\
	\textsc{R:}& Used for -omics data analysis and visualization.\\
	& \small{\href{https://github.com/BradhamLab/scPipe}{https://github.com/BradhamLab/scPipe}} \\
	\textsc{MATLAB:}& Used for numerical analysis of different mathematical systems.\\
	& \small{\href{https://github.com/HawkinsDakota/MCM2015}{https://github.com/HawkinsDakota/MCM2015}}\\
	\textsc{C++:} & Used for computer vision tasks including object detection and segmentation.\\
	& \small{\href{https://github.com/dakota-hawkins/ComputerVision}{https://github.com/dakota-hawkins/ComputerVision}} \\
\end{tabular}
%Section: Posters
\section{\color{linkcolour}{Selected Posters and Presentations}}
\begin{tabular}{rl}
\textsc{2020} & \emph{ICAT: A Novel Method for Identifying Cell-types across
					  Treatments in Single-cell} \\
			  & \emph{RNA Sequencing Data} \\
			  & \textbf{Bioinformatics Open House} \\
			  & \small{Unveiled new algorithm to accurately identify cell-types
			           across biological conditions.} \\
\textsc{2019} & \emph{Subpopulation Discovery During Patterning-Induced
					  Developmental Diversification} \\
			  & \emph{in Sea Urchin Embryos via Single-Cell RNA-Seq} \\
			  & \textbf{Society for Developmental Biology} \\
			  & \small{Presented work showcasing subpopulation disruption
					   during perturbation experiments.} \\
					   
\text{2018} & \emph{Automated Identification of Primary Mesenchyme Cells
					in Confocal Images} \\
			& \textbf{International Conference for the Developmental Biology of
					  the Sea Urchin XXV} \\
			& \small{Presented a computer vision algorithm to identify 3
					 Dimensional cell boundaries.} \\
					 
\text{2017} & \emph{Subpopulation Discovery During Patterning-Induced
					Developmental Diversification} \\
			& \emph{in Sea Urchin Embryos via Single-Cell RNA-Seq} \\
			& \textbf{The International Workshop on Bioinformatics and Systems Biology} \\
			& \small{Presented work identifying novel subpopulations of Primary
					 Mesenchyme Cells during} \\
		    & \small{sea urchin development.} \\

\textsc{2014}& \emph{Detecting Singing on the Nest}
\\& \textbf{Westminster College Undergraduate Research Conference}
\\& \small{Presented undergraduate work to automatically isolate bird songs in
           field recordings.} \\

\textsc{2014}& \emph{An Interdisciplinary Quantitative Analysis and Research Cooperate (QUARC) at} \\
&\emph{Westminster College}
\\ & \textbf{Electronic Conference on Teaching Statistics}
\\ & \small{Helped present current activities and goals of QUARC to promote quantitative reasoning at}
\\ &\small{ Westminster College.} \\

\textsc{2014}& \emph{O Captain! My Captain!}
\\& \textbf{Mathematical Association of America, Intermountain Section}
\\& \small{Presented methods to determine the best college sports coach over
           the past century.}  \\

\textsc{2014} &\emph{Introducing QUARC}
\\& \textbf{Westminster College - Tutorpalooza}
\\& \small{Presented activities and goals of QUARC to fellow tutors and aids on
           Westminster campus.} \\

\textsc{2013}& \emph{Frequency Characteristics of Urban House Finch Songs}
\\& \textbf{Ecological Society of America}
\\& \small{Presented undergraduate research on house finch dialects in urban
           areas within Salt Lake.} \\

\textsc{2013}& \emph{Frequency Characteristics of Urban House Finch Songs}
\\& \textbf{Utah Conference on Undergraduate Research}
\\& \small{Presented undergraduate research on house finch dialects in urban
           areas within Salt Lake.}
\end{tabular}

%Section: Awards and Accolades
\section{\color{linkcolour}{Awards and Accolades}}
\begin{tabular}{rl}
2020 & 1st Place Poster -- Bioinformatics Open House, Boston University \\ 
2017 & 2nd Place Poster -- IBSB Conference, Berlin Germany \\
2016 & NIH Trainee Fellowship -- Boston University \\
2016 & Outstanding Performance Award -- Pacific Northwest National Laboratory \\
2014, 2015 & Honorable Mention -- Mathematical Competition in Modeling \\
2013 -- 2015 &  Gore Math/Science Scholarship -- Wesminster College \\
2013, 2014 & Gore Math/Science Summer Research Grant -- Westminster College \\
2012 & Scholars Summer Research Grant -- Westminster College
\end{tabular}

%Section: Professional Affiliations
\section{\color{linkcolour}{Professional Affiliations}}
\begin{tabular}{rl}
2014 -- Present & Beta Beta Beta (Biology Honor Society) \\
\end{tabular}
\end{document}
